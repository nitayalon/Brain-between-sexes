\documentclass[12pt]{article}
\usepackage{amsmath}
\usepackage{lettrine}
\usepackage{amsfonts}
\usepackage{hyperref}
\usepackage[utf8]{inputenc}
\usepackage[english]{babel}
 \usepackage{bbm}
% \usepackage{tikz}
\usepackage[usenames, dvipsnames]{color}
\newtheorem{remark}{Remark} 
\newtheorem{theorem}{Theorem}[section]
\newtheorem{corollary}{Corollary}[theorem]
\newtheorem{lemma}[theorem]{Lemma}

\newcommand{\transmatrix}[9]{\begin{bmatrix}
 #1&#2&#3\\ #4&#5&#6\\ #7&#8&#9 \end{bmatrix}}

\newcommand{\tocomplete}{[To be completed]}

\newcommand{\addcomment}[1]{\textcolor{red}{#1}}

\newcommand{\addanswer}[1]{\textcolor{OliveGreen}{#1}}


\title{Gaussian Mixture Model}
\author{Nitay Alon}

\begin{document}
\maketitle
We are testing to see if we can use the given sample size (1120) to perform a likelihood ratio test. \\
We assume the following mixture of models:
	\begin{itemize}
		\item Men:
		 \begin{equation*}
		 f_X(x) = p * \mathcal{N}(\theta_{mas},1) + (1-p) * \mathcal{N}(\theta_{fem},1)
		 \end{equation*}
		 \item Women:
		 \begin{equation*}
		 q * \mathcal{N}(\theta_{mas},1) + (1-q) * \mathcal{N}(\theta_{fem},1)
		 \end{equation*}
	\end{itemize}
We assume that: $\theta_{fem} = -\theta_{mas}$ and also $p \approx 1, q \approx 0$.
Denoting $x_i$ as the $i^{th}$ observation from the men group and $y_j$ as the $j^{th}$ observation from the women group and 
using EM algorithm for GMM we can estimate the parameters as follows:
\begin{multline}
M_i = P(z_i = M|x_i) = \frac{p * \mathcal{N}(\theta_{mas},1)}{p * \mathcal{N}(\theta_{mas},1) + (1-p) * \mathcal{N}(\theta_{fem},\sigma)
} = \\
\big(1 + \frac{1-p}{p}exp^{-\frac{(x_i - (\frac{\theta_{mas}+\theta_{fem})}{2}(\theta_{mas}-\theta_{fem})}{2\sigma^2}} \big)^{-1}
\end{multline}
The same formula can be applied to the women data:
\begin{multline}
M_j = P(z_i = M|y_j) = \frac{q * \mathcal{N}(\theta_{mas},1)}{q * \mathcal{N}(\theta_{mas},1) + (1-q) * \mathcal{N}(\theta_{fem},\sigma)
} = \\
\big(1 + \frac{1-q}{q}exp^{-\frac{(y_j - (\frac{\theta_{mas}+\theta_{fem})}{2}(\theta_{mas}-\theta_{fem})}{2\sigma^2}} \big)^{-1}
\end{multline}
And the parameters:
\begin{gather} 	
A = \frac{1}{\sum_{i=1}^{m}M_i}\sum_{i=1}^{m}M_i x_i \\
B = \frac{1}{\sum_{j=1}^{n}M_j}\sum_{j=1}^{n}M_j y_j \\
p = \frac{\sum_{i=1}^{m}M_i}{m} \\
q = \frac{\sum_{j=1}^{n}M_j}{n} \\
\end{gather}
Denoting:
\begin{gather} 	
m_{men} = \bar{x} \\
mm_{men} = \bar{x^2} \\
m_{women} = \bar{y} \\
mm_{women} = \bar{y^2}
\end{gather}
we can estimate the parameters:
\begin{gather} 	
\theta_{mas} = \frac{A + B}{p + q} \\
\theta_{fem} = \frac{m1 + m2 - A - B}{2 - p - q} \\
\sigma^2 = \frac{mm_{mas} + mm_{fem}}{2} - \theta_{fem}^2 + \big(\frac{p + q}{2} \big)(\theta_{fem}^2  - \theta_{mas}^2 )
\end{gather}

\section{Log likelihood ratio} 
Using the EM results we can compute the log likelihood of the mixture model and the log likelihood of the null model	
\begin{gather*}
H_0: p = q \\
H_0: p \neq q 
\end{gather*}
By computing the delta between the llk we can learn about the significance of the delta (recall that $\lambda \sim exp(\theta)$). 

\subsection{30/5/2018 - Update}
After some not satisfying results, we're going over the EM equations again. This time we begin with a constrained version of the problem:
\begin{gather}
p = \frac{\theta_2 + \mu}{\theta_2 + \theta_1} \\
q = \frac{\theta_2 - \mu}{\theta_2 + \theta_1} \\
\sigma^2 = 1 + \theta_1 \theta_2
\end{gather}


\end{document} 